\documentclass[11pt,a4paper,APS]{revtex4}
\usepackage{geometry}
\usepackage{amssymb}
\usepackage{amsmath}
\usepackage{standalone}
\usepackage{preview}
\usepackage{mathtools}
\usepackage{wrapfig}
\usepackage{physics}
\usepackage{simpler-wick}
\usepackage{bm}
\usepackage[colorlinks=true, linkcolor=blue]{hyperref}
%\usepackage[compat=1.1.0]{tikz-feynman}

\geometry{left=2.0cm,right=2.0cm,top=2.5cm,bottom=2.5cm}
\pagestyle{plain}
\begin{document}
\title{The formulism of Standard Model}
%\date{}
\author{Chencan Wang}
\affiliation{School of physics, Nankai university}
\maketitle


	\section{The complex scalar field}
		To incooporate the {\bf electromagnetic interaction} and {\bf weak interaction}, 
		we need an overall description of the intermediate Boson $\gamma$ and $W^{\pm}$,
		$Z^0$. Nevertheless the Bosons for weak interaction, unlike the photon for 
		electromagneitc interaction, are massive particles. We consider the simultaneous 
		symmetry breaking (SSB) mechanism to generate the masses for these bosons, which needs
		at least three Goldenstone boson. 
		The Lagrangian density of the complex scalar field writes 
		\begin{equation}
			\mathcal{L}_\textrm{scalar}=(\partial_\mu \Phi)^\dag (\partial^\mu \Phi)
			-V(\Phi^\dag\Phi),
		\end{equation}
		where the potential term 
		\begin{equation}\label{sec1-phi4pot}
			V(\Phi^\dag\Phi) = -\mu^2 \Phi^\dag\Phi +\lambda (\Phi^\dag\Phi)^2
			\quad \mu>0,~\lambda>0,
		\end{equation}
		with the opposite sign of $\mu^2$ to trigger SSB. 
		The potential~\eqref{sec1-phi4pot} has global minimum values $\langle \Phi^\dag \Phi
		\rangle = \frac{\mu^2}{2\lambda}$, then we denote the vacuum expectation value (VEV)
		\begin{equation}\label{sec1-vev}
			v=\sqrt{\frac{\mu^2}{\lambda}}.
		\end{equation}
		The vacuum state $|0\rangle$ is not the {\it actually lowest energy state} under the 
		background of this potential, we can write the complex field as
		\begin{equation}\label{sec1-complxfield}
			\Phi =\left(\begin{array}{c}
			G^+ \\
			\frac{h+iG_0 + v}{\sqrt{2}} \\ 
			\end{array}\right), \quad 
			\Phi^\dag = \left(G^-,\frac{h-iG_0 + v}{\sqrt{2}}\right).
		\end{equation}
		Given the gauge transformation 
		\begin{equation}
			U = SU(2)_L\times U_Y(1) = \exp\left(-i\beta_a T^a\right) \exp(-i\alpha Y), 
		\end{equation}
		with the $SU(2)_L$ generating element and hyper charge being respectively
		\[
			T^a = \frac{\sigma^a}{2}~~(a=1,~2,~3), \quad Y = \frac{1}{2}.
		\]
		Under the $U$ transformation, the derivative of the transformed field will bring in a 
		new terms in red color 
		\begin{equation}
			\partial_\mu(U\Phi) = U\partial_\mu \Phi + \partial_\mu U \Phi 
			=U[\partial_\mu 
			{\color{red}-i(\partial_\mu\beta_a T^a + \partial_\mu \alpha Y )]\Phi},
		\end{equation}
		which  corresponds to include the interactions 
		\begin{equation}
			-i\partial_\mu \beta_a T^a ~\rightarrow~ g_2 B_\mu^a T_a ,\quad\text{and}
			\quad 
			-i \partial_\mu \alpha Y ~\rightarrow~ g_1 C_\mu Y,
		\end{equation}
		so that the Lagrangian is invariant under $U$ transformation. 
		Thus we can rewrite the derivative in Eq.~\eqref{sec1-complxfield} as the covariant one
		\begin{align}\nonumber 
			D_\mu \Phi&= (\partial_\mu + g_2 T_a B^a_\mu + g_2 YC_\mu) \Phi \\
			\label{sec1-covderiv-pre1}
			& =\left[\partial_\mu + \frac{1}{2}\left(\begin{array}{cc}
			g_2 B_\mu^3 + g_1 C_\mu    &    g_2(B_\mu^1 - i B_\mu^2)\\
			g_2(B_\mu^1 + i B_\mu^2)   &    -g_2B_\mu^3 + g_1 C_\mu \\
			\end{array}\right)\right]\left(\begin{array}{c}
			G^+ \\
			\frac{h+iG_0 + v}{\sqrt{2}} \\ 
			\end{array}\right).
		\end{align}
		In Eq.~\eqref{sec1-covderiv-pre1}, we can further introduce following definitions
		\begin{equation}\label{sec1-defin}
		\begin{gathered}
			W^\pm = \frac{B_\mu^1 \mp B_\mu^2}{\sqrt{2}}, \\
			\left\{\begin{array}{r}
			 	 g_2 B_\mu^3 + g_1 C_\mu = c_{11} Z_\mu^0 + c_{12} A_\mu, \\
			    -g_2 B_\mu^3 + g_1 C_\mu = Z_\mu^0 + c_{22} A_\mu,\\ 
			    \end{array}\right.
		\end{gathered}
		\end{equation}
		with which, we can rewrite Eq.~\eqref{sec1-covderiv-pre1} as 
		\begin{align}\nonumber 
			D_\mu \Phi&= (\partial_\mu + g_2 T_a B^a_\mu + g_2 YC_\mu) \Phi \\
			\label{sec1-covderiv-pre2}
			& =\left[\partial_\mu + \left(\begin{array}{cc}
			\frac{c_{11}}{2} Z_\mu^0 + \frac{c_{12}}{2} A_\mu&\frac{g_2}{\sqrt{2}}W_\mu^+ \\
			\frac{g_2}{\sqrt{2}}W_\mu^- & \frac{c_{21}}{2}Z_\mu^0+\frac{c_{22}}{2}A_\mu\\
			\end{array}\right)\right]\left(\begin{array}{c}
			G^+ \\
			\frac{h+iG_0 + v}{\sqrt{2}} \\ 
			\end{array}\right).
		\end{align}
		Now we focus on the term $(D_\mu\Phi)^\dag(D^\mu \Phi)$, easily one can find, the 
		boson masses can be generated from 
		\begin{equation}
		\begin{aligned}
			&\left(0,\frac{v}{\sqrt{2}}\right)\left(\begin{array}{cc}
			\frac{c_{11}}{2} Z^{0\mu} + \frac{c_{12}}{2} A^\mu&\frac{g_2}{\sqrt{2}}W^{+\mu} \\
			\frac{g_2}{\sqrt{2}}W^{-\mu} & \frac{c_{21}}{2}Z^{0\mu}\\
			\end{array}\right)
			\left(\begin{array}{cc}
			\frac{c_{11}}{2} Z_\mu^0 + \frac{c_{12}}{2} A_\mu&\frac{g_2}{\sqrt{2}}W_\mu^+ \\
			\frac{g_2}{\sqrt{2}}W_\mu^- & \frac{c_{21}}{2}Z_\mu^0\\
			\end{array}\right)
			\left(\begin{array}{c}
			0\\
			\frac{v}{\sqrt{2}} \\ 
			\end{array}\right) \\
			=& 
		\end{aligned}
		\end{equation}  

\end{document}
